\documentclass[a5paper]{scrbook}

% Disable left-right margin on page turn
\KOMAoptions{twoside=false}

% Util Package
\usepackage{bimscommands}
% EB Garamond Font for Holy Writ
\usepackage[oldstyle]{ebgaramond}
\usepackage{fontenc}
% Scripture passages
\usepackage{scripture}
\scripturesetup{font=\ebgaramond}
% Drop Caps
\usepackage{Carrickc,lettrine}
\renewcommand\LettrineFontHook{\Carrickcfamily}

\begin{document}
	\setmainfont{Times}
	\title{The Sorrow of Christ}
	\subtitle{In Old Fashioned English}
	\author{St. Thomas More}
	\frontmatter
	\maketitle
	\tableofcontents
	\mainmatter
	\bimspart{Of The Sorrow, Weariness, Fear, And Prayer Of Christ Before His Taking}{As it is written in the XXVI Chapter of St. Matthew, The XIII of St. Mark, The XXII of St. Luke, and the XVIII of St. John.}
	\chapter{A Meditation on Grace and Olivet}
	
	\begin{scripture}[Matthew 26:30]
		\vs{30}When Jesus had spoken these words, and said grace, they went forth into the
		Mount of Olivet.
	\end{scripture}
	
	\vspace{10mm}
	
	\lettrine{A}{lbeit} that Christ at the time of his supper had had so much godly communication with his apostles, yet forgot he not at his departing to make an end of all together, with thanksgiving to God. But how unlike, alas! be we to Christ, which bear the name of Christian men, and yet at our table do use, not only many vain and idle words (whereof Christ hath given us warning that we shall yield a full strait account), but also very hurtful and perilous, and at last when we have eaten and drunk our fill, unkindly get us our way, forgetting to give thanks unto God the giver of all, that hath so well fed and refreshed us.
	
	Burgensis\footnote{Paul of Burgos (c.1351–1435), a Jewish convert, who later became Archbishop of said Burgos.}, a man well learned and deeply travailed in divinity, upon probable conjectures doth think that the grace, which Christ at the same time said with his apostles, was those six psalms which, as they stand together, the Hebrews call the great Alleluia: that is to wit, the hundredth and twelfth psalm with the five next following in order. For those six psalms, which they name the great Alleluia, they were wont of an old custom to say instead of grace at Easter and certain other high feasts. And the selfsame grace as yet to this day at the said feasts commonly use they to say.
	
	But as for us, whereas we have been accustomed in times past, for grace both before meat and after, to say at sundry seasons sundry psalms such as be most convenient for the time, we have nowadays given them over almost every one, so that with three or four words, whatsoever suddenly cometh to our minds, and those overly mumbled up at adventure, shortly make we an end and depart.
	
	\textscripture[Mt 26:30]{They went forth unto the mount of Olivet.} Forth they went, but not to bed. \textscripture{I rose at midnight,} saith the prophet, \textscripture[Ps 118:62]{to give praise and thanks to thee.} Howbeit Christ did not so much as once lay him down on his bed. But at the leastwise, would God we could truly say: \textscripture[Ps 62:7]{I remembered thee in my bed, good Lord.}
	
	And it was not in the summer season neither that Christ after his supper took his way to the mount. For it was even shortly after the spring of the year, when the days and the nights be all of one length. And that it was a cold night appeareth also by this, that the servants were warming themselves by the fire in the bishop's hall. And that this was not the first time that he so did, well witnesseth the evangelist where he saith: \textscripture{According to his custom.}
	
	He went up to the mount to pray, willing us thereby to understand that when we set ourselves to pray, we must lift up our hearts from the cumbrous unquietness of all worldly business, to the end we may wholly set our minds upon God and godly matters.
	
	This mount of Olivet which was all full of olive trees, containeth in it a certain mystery. For a branch of an olive tree was commonly taken as a token of peace, which Christ came himself to make betwixt God and man, who had so long before been enemies.
	
	Besides this, the oil that cometh of the olive tree doth signify the grace of the Holy Ghost, whom Christ did come to send down to his disciples after his return to his Father: to the end that by the grace of the same Holy Spirit, they might within short space after be able to learn those things which, if he had told them then, they could not well have borne.
	
	\chapter{A Meditation on Cedron}
	
	\begin{scripture}[Jn 18:1, Mt 26:36, Mk 14:32]
		Over a river called Cedron into a village which is named Gethsemani.
	\end{scripture}
	
	\vspace{10mm}
	
	\lettrine{T}{his} river Cedron runneth between the city of Jerusalem and the mount of Olives. And this word Cedron, in the Hebrew tongue, signifieth sorrow or heaviness. And Gethsemani in the same speech is as much to say as a very fat and plentiful valley, or otherwise the valley of Olivet.
	
	We have therefore good cause to think that the evangelists not without great consideration did so diligently rehearse the names of these places, for else they would have thought it sufficient to have shewed that he went forth unto the mount of Olivet, had it not been that God, under the names of those places, had secretly covered some high mysteries, which, by the rehearsal of those names, good men and studious should have occasion afterwards, through the aid of his Holy Spirit, to search out.
	
	For since we may in no wise think that there is any superfluous syllable in the sacred scripture, which the apostles wrote by the inspiration of the Holy Ghost, and that not so much as a sparrow lighteth upon the ground without the will of God\footnote{Matthew 10:29}, I must needs believe that neither the evangelists made mention of those names without some good cause, nor yet that the Hebrews so named them (whatsoever their purpose was when they did so call them), but by some secret motion (albeit to themselves unknown) of God's own Holy Spirit, which under those names had closely hid certain notable mysteries, and at length should be brought to light.
	
	And since Cedron signifieth sorrow and blackness too, and besides that is the name, not of the river only which the evangelists do here make mention of, but also, as we may well perceive, of the valley that the river passeth through, which valley lieth betwixt Jerusalem and Gethsemani, these names (but if we be too slothful and negligent) do put us in remembrance that as long as we live here (as the apostle with), like strangers sequestered from our Lord\footnote{2 Corinthians 5:6}, we must needs pass over, ere ever we come unto the fruitful mount of Olivet, and the pleasant village of Gethsemani (a village, I say, not displeasant or loathsome to look upon, but full of all delight and pleasure), we must first pass over, as I said, this valley and river called Cedron, a vale of misery and river of heaviness, the water whereof may clean, purge, and wash away, the foul black filthiness of our sins.
	
	But now if we, to avoid grief and pain, go about by a contrary way, to make this world, which should be a place of pain and penance, to be a place of ease and pastime, and so turn it unto our heaven, both do we clearly exclude ourselves from the very true felicity for ever, and drown us all too late in fruitless sorrow and care, and further bring ourselves into intolerable and endless wretchedness. And this wholesome lesson are we put in mind of by the wellplaced rehearsal of Cedron and Gethsemani.
	
	Now because the words of holy scripture have not one sense alone, but are full of many mysteries, the names of these places do so well serve to the setting forth of this history of Christ's passion, as though for the same purpose only God had from the beginning ordained those places long before to be called by such notable names, as being compared with those things that Christ did many years after, might declare that they were appointed aforehand to be as it were witnesses of his most bitter passion. For since Cedron signifieth black, doth it not seem to express the saying of the prophet, which was spoken of Christ going to his glorious kingdom by most shameful death, disfigured with stripes, blood, spiteful spitting, and such other filthiness, where it is written: \textscripture[Is 53:2]{Neither comeliness nor beauty is there in him.} And that the river which he passed over did not without cause betoken sorrow and heaviness, himself right well witnessed where he said: \textscripture[Mt 26:38]{My soul is heavy even to the death.}
	
	\chapter{A Meditation on Betrayal}
	
	\begin{scripture}[Luke 22:39]
		And his disciples went with him.
	\end{scripture}
	
	\vspace{10mm}
	
	\lettrine{I}{t} is to be understood of the eleven only which still remained with him. For the twelfth, whom the devil entered into after he had eaten the sop, and carried forth from the residue of the apostles, waited now no longer upon his master as his disciple, but like a traitor laboured to destroy him. And so proved these words of Christ too true: \textscripture[Mt 12:30]{He that is not with me is against me.}
	
	Let us follow Christ therefore, and by prayer call upon his Father with him. And let us not, as Judas did, slip aside from him, after we have been relieved by his gracious goodness, and well and liberally supped with him, for fear this saying of the prophet be verified in us: \textscripture[Ps 49:18]{If thou sawest a thief thou didst run with him, and with adulterers didst thou pay thy shot.}
	
	\chapter{A Meditation on Prayer}
	
	\begin{scripture}[John 18:2]
		And Judas that did go about to betray him, knew right well the place, because Jesus used often times to come thither with his disciples.
	\end{scripture}
	
	\vspace{10mm}
	
	\lettrine{N}{ow} by occasion of the traitor do the evangelists yet once again both beat into us, and with oft rehearsal thereof much commend also, the blessed custom of Christ who was wont to resort thither with his disciples to pray. For if he had not gone to the same place so commonly in the night time, but now and then among, the traitor could not have been so well assured to find our Lord there, that he durst have conducted thither the bishop's servants and a band of the Roman soldiers, as to the thing they should not miss to meet withal; since if they had found it otherwise, they would have went he had mocked them, and so ere he could have escaped away, haply have done him some displeasure.
	
	But now where are these folk become, that stand very much in their own conceit, and as though they had done a great feat, fondly glory in themselves, if it hath fortuned them at one time or other, on high evens, either to watch anything long in prayer by night, or else for the same purpose to rise in the morning somewhat early? Our saviour Christ customably used to persevere in prayer all the whole night without any sleep at all.
	
	Where be they also which, because he refused not to eat and drink with the publicans, nor disdained not to receive kindness and service of sinners, called him a glutton and a drunkard, and in
	comparison of the Pharisees, whose profession was very strait counted him to be scant in virtue so perfect as one of the common sort? And yet while these sour lowering Pharisees, to be seen of the world, were praying openly abroad in corners of the streets, he therewhiles full mildly and lovingly taught sinful men, while he ate and drank with them, to amend their lives. Again while the false dissembling pharisee lay at his ease routing in his soft bed, Christ continued without doors painfully all night in prayer
	
	Oh, would God we which are so slack and slothful that we cannot follow the good example of our saviour in this behalf, would yet at the least wise, when we turn ourselves in our bed even ready to fall asleep, have in remembrance Christ's continual watch, and although it were in few words, till sleep come on us again, give him hearty thanks, both misliking our own sluggishness and therewithal desiring him to endue us with more of his grace. Surely if we would accustom ourselves to do but even so much, I nothing doubt but that God would within short space help us with his grace and make us much better.
	
	\chapter{A Meditation on the Humanity of Christ}
	
	\begin{scripture}[Mt 26:36-38, Mk 14:32-34]
		‘And sit you here,’ quoth he, ‘whiles I go yonder and pray. Then took he Peter with him, and the two sons of Zebedee, and began to be heavy and sad, and to wax somewhat afraid and weary. Then said he unto them: My soul is heavy even unto the death. Abide ye here and watch with me.’
	\end{scripture}
	
	\vspace{10mm}
	
	\lettrine{W}{hereas} Christ willed the other eight of his disciples to stay somewhat behind him, Peter, John, and his brother James caused he to go further with him, as those whom he had always used more familiarly than all the rest of his apostles. Which thing although he had done for none other respect but only for that it liked him so to do, no cause yet had any man to be grieved therewith to see him so good and gracious. Howbeit great considerations were there besides, which as it seemeth moved him thereunto. Forasmuch as Peter for the fervour of his faith, John for his virginity, and his brother James for that he was the first of his apostles that should suffer martyrdom for his sake, did indeed far pass and surmount all the rest. And these three also had he long erst vouchsafed to admit both to be privy to his glorious transfiguration, and also presently to see it. Convenient was it therefore that they whom he had afore all other called with him to so wonderful a sight, and there had comforted for the while with the clear light of his eternal glory, convenient was it, I say, that these three in especial, who as reason would were more strong hearted than the other, should be placed nearest about him at the time of his painful pangs foregoing his bitter passion.
	
	Now when he was gone a little beyond them, straightways he felt himself oppressed with such an horrible heaviness, sorrow, fear, and weariness, and that with so great extremity that by and by even before them, he letted not to utter these lamentable words, that evidently declared the marvellous inward anguish of his sore troubled heart.
	
	\textscripture[Mt 26:38, Mk 14:34]{My soul is heavy even to the death.} For the blessed and tender heart of our most holy saviour was cumbered and panged with manifold and hideous griefs, since doubtless well wist he, that the false traitor and his mortal enemies drew near unto him, and were now in manner already come upon him; and over this that he should be despitefully bounden, and have heinous crimes surmised against him, be blasphemed, scourged, crowned with thorns, nailed, crucified, and finally suffer very long and cruel torments. Moreover much did it disquiet him, that he foresaw the fear and dread which his disciples should fall in, the mischief that should light on the Jews, the destruction of the false traitor Judas, and last of all, the unspeakable sorrow of his dear beloved mother. The storms and heaps of so many troubles coming upon him all at once, as doth the main sea when it violently breaketh down the banks over the land, sore oppressed his most holy and blessed heart.
	
	Some man may haply here marvel how this could be, that our saviour Christ, being very God equal with his almighty Father, could be heavy, sad, and sorrowful. Indeed, he could not have been so, if as he was God, so had he been only God, and not man also. But now seeing he was as verily man as he was verily God, I think it no more to be marvelled that inasmuch as he was man he had these affections and conditions in him, such I mean as be without offence to God, as of common course are in mankind, than that inasmuch as he was God he wrought so wonderful miracles. For if we do marvel that Christ should have in him fear, weariness, and sorrow, namely seeing he was God, then why should we not as well marvel that he was hungry, athirst, and slept, since albeit he had these properties, yet was he nevertheless God for all that? But hereunto peradventure mayst thou reply and say: albeit I do now marvel no more that he could so do, yet can I not but marvel still why he would so do. For what reason is it that he which taught his disciples\footnote{Matthew 10:28} in no wise to fear those that could but kill only their bodies, and when that was done had no further thing in their power wherewith they could do them harm, should now wax afraid of them himself, namely since against his blessed body they could no more do, than it liked his holy majesty to permit and suffer them?
	
	Over this seeing (hereof we be well assured), that his martyrs joyfully and courageously hasted them toward their death, not letting even then boldly to rebuke and reprove the tyrants and their cruel tormentors, how unseemly might it be thought that Christ himself being, as a man might say, the chief bannerbearer and captain of all martyrs, should, when he drew near to his passion, be so sore afraid, so heavy, so wonderfully unquieted and troubled. Had it not been meet that he which did all things himself before he taught the same, should in this point especially in his own person, have given other men example to learn of him, for the truth's sake cheerfully to suffer death; lest such as in time to come would be loath and afraid to die for the defence of the faith, might happly, to excuse their own faint and feeble hearts, bear themselves in hand, that they did none otherwise therein than Christ had done before them. And so doing yet should they both not a little dishonour so good and worthy a master, and besides that much discourage other folk, to see them in so great fear and heaviness.
	
	They that make these objections, and such other like, neither do thoroughly perceive the whole bottom of this matter, nor yet well weigh what Christ's meaning was, when he commanded his disciples in no wise to be afraid of death. For he meant not that they should in no case once shrink at death, but that they should not so shrink and flee from temporal death, that by forsaking the faith, they should fall into endless death for ever. Who though he would have his soldiers to be bold and therewithal discreet, requireth not yet to have them neither like blocks nor madmen. For as he hath a strong courageous heart that never shrinketh patiently to suffer pain, so he that feeleth none, is like a very block without any sense at all. It were a mad part for a man not to fear to have his flesh cut, and yet should no wise man for any dread of pain be withdrawn from his godly purpose, and so, by the refusal of a small pain, purchase himself a much greater.
	
	A surgeon when a diseased place must be lanced or seared, exhorteth not his patient to imagine that at the same time he shall feel no grief or pain at all, but willeth him in any wise quietly to take it. He denieth not but that it will be right painful unto him. But then again the pleasure that he shall have by the recovery of his health and the avoiding of sorer grief likely to ensue, this shall fully, saith he, recompense altogether.
	
	And albeit our saviour Christ biddeth us rather willingly to suffer death, when there is none other remedy, than for fear thereof to forsake him (and forsake him do we, if before the world we refuse to confess his faith),\footnote{Matthew 10:32-33} yet doth he not for all that, command us so to strive against nature, as not once to shrink at death. Insomuch that he giveth us free liberty to avoid all trouble and danger, in case we may so do without prejudice and hindrance of the cause. \textscripture{For if they persecute ye,} saith he, \textscripture[Mt 10:23]{in one city get ye into another.} Upon which merciful licence and provident advice of our most prudent master, none of the apostles was there in a manner, no nor but few of the most notable martyrs neither that suffered so many years after, but that at one time or other they thus preserved their lives; and to the manifold profit both of themselves and many other more, reserved the same until such season as the secret providence of God foresaw to be more convenient.
	
	Howbeit some time Christ's valiant champions have done far otherwise, and of their own accord professed themselves Christian men, when no creature required it of them, and of their own minds, offered their bodies to martyrdom when no man called for them. Thus hath it liked God for the advancement of his honour, some whiles to keep from the knowledge of the world the great abundant faith of his servants, thereby to disappoint their wily and malicious enemies; and some whiles again so to set it forth, that their cruel persecutors were therewith much incensed, while both they saw themselves deceived of their expectation, and were moreover right angry to consider that the martyrs, that offered themselves to die for Christ's sake, could be overcome by no kind of cruelty.
	
	But yet lo! God of his infinite mercy doth not require us to take upon us this most high degree of stout courage which is so full of hardness and difficulty. And therefore I would not advise every man at adventure rashly to run forth so far forward, that he shall not be able fair and softly to come back again, but unless he can attain to climb up the hilltop, be haply in hazard to tumble down even to the bottom headlong. Let them yet whom God especially calleth thereunto, set forth in God's name and proceed, and they shall reign. The times, yea the very instants ofttimes and the causes of all things, hath he secret unto himself, and when he seeth time convenient he doth all things, as his deep wisdom, which pierceth all things mightily and disposeth all things pleasantly,\footnote{Wisdom 8:1} before had secretly determined. Whosoever therefore is brought to such a strait, that needs he must either endure some pain in his body, or else forsake God, this man may be right well assured, that he is by God's own will come to such distress. Whereupon hath he without doubt great occasion to be of good comfort, since either God will not fail to deliver him therefrom again, or be ready at his elbow to assist him in his conflict, and so give him the upper hand, that for his victory shall he be crowned.
	
	\textscripture{For God is true of his promise,} saith the apostle, \textscripture[2 Co 10:13]{who will not suffer ye to be tempted above that ye may bear, but make ye also with the temptation a way out, that ye may have strength to abide it.} Wherefore when we are come to the point, that we must of necessity fight hand to hand with the prince of this world, the devil, and his cruel ministers, so that we cannot shrink back without the defacing of our cause, then would I, lo! counsel every man in this case utterly to cast away all fear. And here would I bid him quietly to set his heart at rest, in the sure hope and trust of God's help, namely seeing the scripture telleth us, that whosoever putteth not his confidence in God in the time of tribulation shall find his strength full feeble. But yet before a man falleth in trouble, fear is not greatly to be discommended; and so that reason be always ready to resist and master fear, the conflict is then no sin nor offence at all, but rather a great matter of merit.
	
\end{document}
