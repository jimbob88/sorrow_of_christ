\documentclass[a5paper]{scrbook}

% Disable left-right margin on page turn
\KOMAoptions{twoside=false}

% Util Package
\usepackage{bimscommands}
% EB Garamond Font for Holy Writ
\usepackage[oldstyle]{ebgaramond}
\usepackage{fontenc}
% Scripture passages
\usepackage{scripture}
\scripturesetup{font=\ebgaramond}
% Drop Caps
\usepackage{Carrickc,lettrine}
\renewcommand\LettrineFontHook{\Carrickcfamily}

\begin{document}
	\setmainfont{Times}
	\title{The Sorrow of Christ}
	\subtitle{In Old Fashioned English}
	\author{St. Thomas More}
	\frontmatter
	\maketitle
	\tableofcontents
	\mainmatter
	\bimspart{Of The Sorrow, Weariness, Fear, And Prayer Of Christ Before His Taking}{As it is written in the XXVI Chapter of St. Matthew, The XIII of St. Mark, The XXII of St. Luke, and the XVIII of St. John.}
	\chapter{A Meditation on Grace and Olivet}
	
	\begin{scripture}[Matthew 26:30]
		\vs{30}When Jesus had spoken these words, and said grace, they went forth into the
		Mount of Olivet.
	\end{scripture}
	
	\vspace{10mm}
	
	\lettrine{A}{lbeit} that Christ at the time of his supper had had so much godly communication with his apostles, yet forgot he not at his departing to make an end of all together, with thanksgiving to God. But how unlike, alas! be we to Christ, which bear the name of Christian men, and yet at our table do use, not only many vain and idle words (whereof Christ hath given us warning that we shall yield a full strait account), but also very hurtful and perilous, and at last when we have eaten and drunk our fill, unkindly get us our way, forgetting to give thanks unto God the giver of all, that hath so well fed and refreshed us.
	
	Burgensis\footnote{Paul of Burgos (c.1351–1435), a Jewish convert, who later became Archbishop of said Burgos.}, a man well learned and deeply travailed in divinity, upon probable conjectures doth think that the grace, which Christ at the same time said with his apostles, was those six psalms which, as they stand together, the Hebrews call the great Alleluia: that is to wit, the hundredth and twelfth psalm with the five next following in order. For those six psalms, which they name the great Alleluia, they were wont of an old custom to say instead of grace at Easter and certain other high feasts. And the selfsame grace as yet to this day at the said feasts commonly use they to say.
	
	But as for us, whereas we have been accustomed in times past, for grace both before meat and after, to say at sundry seasons sundry psalms such as be most convenient for the time, we have nowadays given them over almost every one, so that with three or four words, whatsoever suddenly cometh to our minds, and those overly mumbled up at adventure, shortly make we an end and depart.
	
	\vspace{10mm}
	
	\textscripture[Matthew 26:30]{They went forth unto the mount of Olivet.}
	
	Forth they went, but not to bed. \textscripture{I rose at midnight,} saith the prophet, \textscripture[Ps 118:62]{to give praise and thanks to thee.} Howbeit Christ did not so much as once lay him down on his bed. But at the leastwise, would God we could truly say: \textscripture[Ps 62:7]{I remembered thee in my bed, good Lord.}
	
	And it was not in the summer season neither that Christ after his supper took his way to the mount. For it was even shortly after the spring of the year, when the days and the nights be all of one length. And that it was a cold night appeareth also by this, that the servants were warming themselves by the fire in the bishop's hall. And that this was not the first time that he so did, well witnesseth the evangelist where he saith: \textscripture{According to his custom.}
	
	He went up to the mount to pray, willing us thereby to understand that when we set ourselves to
	pray, we must lift up our hearts from the cumbrous unquietness of all worldly business, to the end we
	may wholly set our minds upon God and godly matters.
	
	This mount of Olivet which was all full of olive trees, containeth in it a certain mystery. For a
	branch of an olive tree was commonly taken as a token of peace, which Christ came himself to make betwixt God and man, who had so long before been enemies.
	
	Besides this, the oil that cometh of the olive tree doth signify the grace of the Holy Ghost, whom
	Christ did come to send down to his disciples after his return to his Father: to the end that by the grace
	of the same Holy Spirit, they might within short space after be able to learn those things which, if he had
	told them then, they could not well have borne.
	
	\chapter{A Meditation on Cedron}
	
\end{document}
