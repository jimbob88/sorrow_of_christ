\documentclass[a5paper]{scrbook}

% Disable left-right margin on page turn
\KOMAoptions{twoside=false}

% Util Package
\usepackage{bimscommands}
% Manage fonts
\usepackage{fontspec}
% Libertine as the main font
\usepackage{libertine}
% EB Garamond Font for Holy Writ
\usepackage[oldstyle]{ebgaramond}
\usepackage{fontenc}
% Scripture passages
\usepackage{scripture}
\scripturesetup{font=\ebgaramond}
% Drop Caps
\usepackage{Carrickc,lettrine}
\renewcommand\LettrineFontHook{\Carrickcfamily}

\begin{document}
	\setmainfont{Linux Libertine O}
	\title{The Sorrow of Christ}
	\subtitle{In Old Fashioned English}
	\author{St. Thomas More}
	\frontmatter
	\maketitle
	\tableofcontents
	\mainmatter
	\bimspart{Of The Sorrow, Weariness, Fear, And Prayer Of Christ Before His Taking}{As it is written in the XXVI Chapter of St. Matthew, The XIII of St. Mark, The XXII of St. Luke, and the XVIII of St. John.}
	\chapter{A Meditation on Grace and Olivet}
	
	\begin{scripture}[Matthew 26:30]
		\vs{30}When Jesus had spoken these words, and said grace, they went forth into the
		Mount of Olivet.
	\end{scripture}
	
	\vspace{10mm}
	
	\lettrine{A}{lbeit} that Christ at the time of his supper had had so much godly communication with his apostles, yet forgot he not at his departing to make an end of all together, with thanksgiving to God. But how unlike, alas! be we to Christ, which bear the name of Christian men, and yet at our table do use, not only many vain and idle words (whereof Christ hath given us warning that we shall yield a full strait account), but also very hurtful and perilous, and at last when we have eaten and drunk our fill, unkindly get us our way, forgetting to give thanks unto God the giver of all, that hath so well fed and refreshed us.
	
	Burgensis\footnote{Paul of Burgos (c.1351–1435), a Jewish convert, who later became Archbishop of said Burgos.}, a man well learned and deeply travailed in divinity, upon probable conjectures doth think that the grace, which Christ at the same time said with his apostles, was those six psalms which, as they stand together, the Hebrews call the great Alleluia: that is to wit, the hundredth and twelfth psalm with the five next following in order. For those six psalms, which they name the great Alleluia, they were wont of an old custom to say instead of grace at Easter and certain other high feasts. And the selfsame grace as yet to this day at the said feasts commonly use they to say.
	
	But as for us, whereas we have been accustomed in times past, for grace both before meat and after, to say at sundry seasons sundry psalms such as be most convenient for the time, we have nowadays given them over almost every one, so that with three or four words, whatsoever suddenly cometh to our minds, and those overly mumbled up at adventure, shortly make we an end and depart.
	
	\textscripture[Mt 26:30]{They went forth unto the mount of Olivet.} Forth they went, but not to bed. \textscripture{I rose at midnight,} saith the prophet, \textscripture[Ps 118:62]{to give praise and thanks to thee.} Howbeit Christ did not so much as once lay him down on his bed. But at the leastwise, would God we could truly say: \textscripture[Ps 62:7]{I remembered thee in my bed, good Lord.}
	
	And it was not in the summer season neither that Christ after his supper took his way to the mount. For it was even shortly after the spring of the year, when the days and the nights be all of one length. And that it was a cold night appeareth also by this, that the servants were warming themselves by the fire in the bishop's hall. And that this was not the first time that he so did, well witnesseth the evangelist where he saith: \textscripture{According to his custom.}
	
	He went up to the mount to pray, willing us thereby to understand that when we set ourselves to pray, we must lift up our hearts from the cumbrous unquietness of all worldly business, to the end we may wholly set our minds upon God and godly matters.
	
	This mount of Olivet which was all full of olive trees, containeth in it a certain mystery. For a branch of an olive tree was commonly taken as a token of peace, which Christ came himself to make betwixt God and man, who had so long before been enemies.
	
	Besides this, the oil that cometh of the olive tree doth signify the grace of the Holy Ghost, whom Christ did come to send down to his disciples after his return to his Father: to the end that by the grace of the same Holy Spirit, they might within short space after be able to learn those things which, if he had told them then, they could not well have borne.
	
	\chapter{A Meditation on Cedron}
	
	\begin{scripture}[Jn 18:1, Mt 26:36, Mk 14:32]
		Over a river called Cedron into a village which is named Gethsemani.
	\end{scripture}
	
	\vspace{10mm}
	
	\lettrine{T}{his} river Cedron runneth between the city of Jerusalem and the mount of Olives. And this word Cedron, in the Hebrew tongue, signifieth sorrow or heaviness. And Gethsemani in the same speech is as much to say as a very fat and plentiful valley, or otherwise the valley of Olivet.
	
	We have therefore good cause to think that the evangelists not without great consideration did so diligently rehearse the names of these places, for else they would have thought it sufficient to have shewed that he went forth unto the mount of Olivet, had it not been that God, under the names of those places, had secretly covered some high mysteries, which, by the rehearsal of those names, good men and studious should have occasion afterwards, through the aid of his Holy Spirit, to search out.
	
	For since we may in no wise think that there is any superfluous syllable in the sacred scripture, which the apostles wrote by the inspiration of the Holy Ghost, and that not so much as a sparrow lighteth upon the ground without the will of God\footnote{Matthew 10:29}, I must needs believe that neither the evangelists made mention of those names without some good cause, nor yet that the Hebrews so named them (whatsoever their purpose was when they did so call them), but by some secret motion (albeit to themselves unknown) of God's own Holy Spirit, which under those names had closely hid certain notable mysteries, and at length should be brought to light.
	
	And since Cedron signifieth sorrow and blackness too, and besides that is the name, not of the river only which the evangelists do here make mention of, but also, as we may well perceive, of the valley that the river passeth through, which valley lieth betwixt Jerusalem and Gethsemani, these names (but if we be too slothful and negligent) do put us in remembrance that as long as we live here (as the apostle with), like strangers sequestered from our Lord\footnote{2 Corinthians 5:6}, we must needs pass over, ere ever we come unto the fruitful mount of Olivet, and the pleasant village of Gethsemani (a village, I say, not displeasant or loathsome to look upon, but full of all delight and pleasure), we must first pass over, as I said, this valley and river called Cedron, a vale of misery and river of heaviness, the water whereof may clean, purge, and wash away, the foul black filthiness of our sins.
	
	But now if we, to avoid grief and pain, go about by a contrary way, to make this world, which should be a place of pain and penance, to be a place of ease and pastime, and so turn it unto our heaven, both do we clearly exclude ourselves from the very true felicity for ever, and drown us all too late in fruitless sorrow and care, and further bring ourselves into intolerable and endless wretchedness. And this wholesome lesson are we put in mind of by the wellplaced rehearsal of Cedron and Gethsemani.
	
	Now because the words of holy scripture have not one sense alone, but are full of many mysteries, the names of these places do so well serve to the setting forth of this history of Christ's passion, as though for the same purpose only God had from the beginning ordained those places long before to be called by such notable names, as being compared with those things that Christ did many years after, might declare that they were appointed aforehand to be as it were witnesses of his most bitter passion. For since Cedron signifieth black, doth it not seem to express the saying of the prophet, which was spoken of Christ going to his glorious kingdom by most shameful death, disfigured with stripes, blood, spiteful spitting, and such other filthiness, where it is written: \textscripture[Is 53:2]{Neither comeliness nor beauty is there in him.} And that the river which he passed over did not without cause betoken sorrow and heaviness, himself right well witnessed where he said: \textscripture[Mt 26:38]{My soul is heavy even to the death.}
	
	\chapter{A Meditation on Betrayal}
	
	\begin{scripture}[Luke 22:39]
		And his disciples went with him.
	\end{scripture}
	
	\vspace{10mm}
	
	\lettrine{I}{t} is to be understood of the eleven only which still remained with him. For the twelfth, whom the devil entered into after he had eaten the sop, and carried forth from the residue of the apostles, waited now no longer upon his master as his disciple, but like a traitor laboured to destroy him. And so proved these words of Christ too true: \textscripture[Mt 12:30]{He that is not with me is against me.}
	
	Let us follow Christ therefore, and by prayer call upon his Father with him. And let us not, as Judas did, slip aside from him, after we have been relieved by his gracious goodness, and well and liberally supped with him, for fear this saying of the prophet be verified in us: \textscripture[Ps 49:18]{If thou sawest a thief thou didst run with him, and with adulterers didst thou pay thy shot.}
	
	\chapter{A Meditation on Prayer}
	
	\begin{scripture}[John 18:2]
		And Judas that did go about to betray him, knew right well the place, because Jesus used often times to come thither with his disciples.
	\end{scripture}
	
	\vspace{10mm}
	
	\lettrine{N}{ow} by occasion of the traitor do the evangelists yet once again both beat into us, and with oft rehearsal thereof much commend also, the blessed custom of Christ who was wont to resort thither with his disciples to pray. For if he had not gone to the same place so commonly in the night time, but now and then among, the traitor could not have been so well assured to find our Lord there, that he durst have conducted thither the bishop's servants and a band of the Roman soldiers, as to the thing they should not miss to meet withal; since if they had found it otherwise, they would have went he had mocked them, and so ere he could have escaped away, haply have done him some displeasure.
	
	But now where are these folk become, that stand very much in their own conceit, and as though they had done a great feat, fondly glory in themselves, if it hath fortuned them at one time or other, on high evens, either to watch anything long in prayer by night, or else for the same purpose to rise in the morning somewhat early? Our saviour Christ customably used to persevere in prayer all the whole night without any sleep at all.
	
	Where be they also which, because he refused not to eat and drink with the publicans, nor disdained not to receive kindness and service of sinners, called him a glutton and a drunkard, and in
	comparison of the Pharisees, whose profession was very strait counted him to be scant in virtue so perfect as one of the common sort? And yet while these sour lowering Pharisees, to be seen of the world, were praying openly abroad in corners of the streets, he therewhiles full mildly and lovingly taught sinful men, while he ate and drank with them, to amend their lives. Again while the false dissembling pharisee lay at his ease routing in his soft bed, Christ continued without doors painfully all night in prayer
	
	Oh, would God we which are so slack and slothful that we cannot follow the good example of our saviour in this behalf, would yet at the least wise, when we turn ourselves in our bed even ready to fall asleep, have in remembrance Christ's continual watch, and although it were in few words, till sleep come on us again, give him hearty thanks, both misliking our own sluggishness and therewithal desiring him to endue us with more of his grace. Surely if we would accustom ourselves to do but even so much, I nothing doubt but that God would within short space help us with his grace and make us much better.
	
	\chapter{A Meditation on the Humanity of Christ}
	
	\begin{scripture}[Mt 26:36-38, Mk 14:32-34]
		‘And sit you here,’ quoth he, ‘whiles I go yonder and pray. Then took he Peter with him, and the two sons of Zebedee, and began to be heavy and sad, and to wax somewhat afraid and weary. Then said he unto them: My soul is heavy even unto the death. Abide ye here and watch with me.’
	\end{scripture}
	
	\vspace{10mm}
	
	\lettrine{W}{hereas} Christ willed the other eight of his disciples to stay somewhat behind him, Peter, John, and his brother James caused he to go further with him, as those whom he had always used more familiarly than all the rest of his apostles. Which thing although he had done for none other respect but only for that it liked him so to do, no cause yet had any man to be grieved therewith to see him so good and gracious. Howbeit great considerations were there besides, which as it seemeth moved him thereunto. Forasmuch as Peter for the fervour of his faith, John for his virginity, and his brother James for that he was the first of his apostles that should suffer martyrdom for his sake, did indeed far pass and surmount all the rest. And these three also had he long erst vouchsafed to admit both to be privy to his glorious transfiguration, and also presently to see it. Convenient was it therefore that they whom he had afore all other called with him to so wonderful a sight, and there had comforted for the while with the clear light of his eternal glory, convenient was it, I say, that these three in especial, who as reason would were more strong hearted than the other, should be placed nearest about him at the time of his painful pangs foregoing his bitter passion.
	
	Now when he was gone a little beyond them, straightways he felt himself oppressed with such an horrible heaviness, sorrow, fear, and weariness, and that with so great extremity that by and by even before them, he letted not to utter these lamentable words, that evidently declared the marvellous inward anguish of his sore troubled heart.
	
	\textscripture[Mt 26:38, Mk 14:34]{My soul is heavy even to the death.} For the blessed and tender heart of our most holy saviour was cumbered and panged with manifold and hideous griefs, since doubtless well wist he, that the false traitor and his mortal enemies drew near unto him, and were now in manner already come upon him; and over this that he should be despitefully bounden, and have heinous crimes surmised against him, be blasphemed, scourged, crowned with thorns, nailed, crucified, and finally suffer very long and cruel torments. Moreover much did it disquiet him, that he foresaw the fear and dread which his disciples should fall in, the mischief that should light on the Jews, the destruction of the false traitor Judas, and last of all, the unspeakable sorrow of his dear beloved mother. The storms and heaps of so many troubles coming upon him all at once, as doth the main sea when it violently breaketh down the banks over the land, sore oppressed his most holy and blessed heart.
	
	Some man may haply here marvel how this could be, that our saviour Christ, being very God equal with his almighty Father, could be heavy, sad, and sorrowful. Indeed, he could not have been so, if as he was God, so had he been only God, and not man also. But now seeing he was as verily man as he was verily God, I think it no more to be marvelled that inasmuch as he was man he had these affections and conditions in him, such I mean as be without offence to God, as of common course are in mankind, than that inasmuch as he was God he wrought so wonderful miracles. For if we do marvel that Christ should have in him fear, weariness, and sorrow, namely seeing he was God, then why should we not as well marvel that he was hungry, athirst, and slept, since albeit he had these properties, yet was he nevertheless God for all that? But hereunto peradventure mayst thou reply and say: albeit I do now marvel no more that he could so do, yet can I not but marvel still why he would so do. For what reason is it that he which taught his disciples\footnote{Matthew 10:28} in no wise to fear those that could but kill only their bodies, and when that was done had no further thing in their power wherewith they could do them harm, should now wax afraid of them himself, namely since against his blessed body they could no more do, than it liked his holy majesty to permit and suffer them?
	
	Over this seeing (hereof we be well assured), that his martyrs joyfully and courageously hasted them toward their death, not letting even then boldly to rebuke and reprove the tyrants and their cruel tormentors, how unseemly might it be thought that Christ himself being, as a man might say, the chief bannerbearer and captain of all martyrs, should, when he drew near to his passion, be so sore afraid, so heavy, so wonderfully unquieted and troubled. Had it not been meet that he which did all things himself before he taught the same, should in this point especially in his own person, have given other men example to learn of him, for the truth's sake cheerfully to suffer death; lest such as in time to come would be loath and afraid to die for the defence of the faith, might happly, to excuse their own faint and feeble hearts, bear themselves in hand, that they did none otherwise therein than Christ had done before them. And so doing yet should they both not a little dishonour so good and worthy a master, and besides that much discourage other folk, to see them in so great fear and heaviness.
	
	They that make these objections, and such other like, neither do thoroughly perceive the whole bottom of this matter, nor yet well weigh what Christ's meaning was, when he commanded his disciples in no wise to be afraid of death. For he meant not that they should in no case once shrink at death, but that they should not so shrink and flee from temporal death, that by forsaking the faith, they should fall into endless death for ever. Who though he would have his soldiers to be bold and therewithal discreet, requireth not yet to have them neither like blocks nor madmen. For as he hath a strong courageous heart that never shrinketh patiently to suffer pain, so he that feeleth none, is like a very block without any sense at all. It were a mad part for a man not to fear to have his flesh cut, and yet should no wise man for any dread of pain be withdrawn from his godly purpose, and so, by the refusal of a small pain, purchase himself a much greater.
	
	A surgeon when a diseased place must be lanced or seared, exhorteth not his patient to imagine that at the same time he shall feel no grief or pain at all, but willeth him in any wise quietly to take it. He denieth not but that it will be right painful unto him. But then again the pleasure that he shall have by the recovery of his health and the avoiding of sorer grief likely to ensue, this shall fully, saith he, recompense altogether.
	
	And albeit our saviour Christ biddeth us rather willingly to suffer death, when there is none other remedy, than for fear thereof to forsake him (and forsake him do we, if before the world we refuse to confess his faith),\footnote{Matthew 10:32-33} yet doth he not for all that, command us so to strive against nature, as not once to shrink at death. Insomuch that he giveth us free liberty to avoid all trouble and danger, in case we may so do without prejudice and hindrance of the cause. \textscripture{For if they persecute ye,} saith he, \textscripture[Mt 10:23]{in one city get ye into another.} Upon which merciful licence and provident advice of our most prudent master, none of the apostles was there in a manner, no nor but few of the most notable martyrs neither that suffered so many years after, but that at one time or other they thus preserved their lives; and to the manifold profit both of themselves and many other more, reserved the same until such season as the secret providence of God foresaw to be more convenient.
	
	Howbeit some time Christ's valiant champions have done far otherwise, and of their own accord professed themselves Christian men, when no creature required it of them, and of their own minds, offered their bodies to martyrdom when no man called for them. Thus hath it liked God for the advancement of his honour, some whiles to keep from the knowledge of the world the great abundant faith of his servants, thereby to disappoint their wily and malicious enemies; and some whiles again so to set it forth, that their cruel persecutors were therewith much incensed, while both they saw themselves deceived of their expectation, and were moreover right angry to consider that the martyrs, that offered themselves to die for Christ's sake, could be overcome by no kind of cruelty.
	
	But yet lo! God of his infinite mercy doth not require us to take upon us this most high degree of stout courage which is so full of hardness and difficulty. And therefore I would not advise every man at adventure rashly to run forth so far forward, that he shall not be able fair and softly to come back again, but unless he can attain to climb up the hilltop, be haply in hazard to tumble down even to the bottom headlong. Let them yet whom God especially calleth thereunto, set forth in God's name and proceed, and they shall reign. The times, yea the very instants ofttimes and the causes of all things, hath he secret unto himself, and when he seeth time convenient he doth all things, as his deep wisdom, which pierceth all things mightily and disposeth all things pleasantly,\footnote{Wisdom 8:1} before had secretly determined. Whosoever therefore is brought to such a strait, that needs he must either endure some pain in his body, or else forsake God, this man may be right well assured, that he is by God's own will come to such distress. Whereupon hath he without doubt great occasion to be of good comfort, since either God will not fail to deliver him therefrom again, or be ready at his elbow to assist him in his conflict, and so give him the upper hand, that for his victory shall he be crowned.
	
	\textscripture{For God is true of his promise,} saith the apostle, \textscripture[2 Co 10:13]{who will not suffer ye to be tempted above that ye may bear, but make ye also with the temptation a way out, that ye may have strength to abide it.} Wherefore when we are come to the point, that we must of necessity fight hand to hand with the prince of this world, the devil, and his cruel ministers, so that we cannot shrink back without the defacing of our cause, then would I, lo! counsel every man in this case utterly to cast away all fear. And here would I bid him quietly to set his heart at rest, in the sure hope and trust of God's help, namely seeing the scripture telleth us, that whosoever putteth not his confidence in God in the time of tribulation shall find his strength full feeble. But yet before a man falleth in trouble, fear is not greatly to be discommended; and so that reason be always ready to resist and master fear, the conflict is then no sin nor offence at all, but rather a great matter of merit.
	
	For weenest thou that those holy martyrs which shed their blood for Christ's faith were never afraid of death and pain? I will not spend much time in this behalf to make any long rehearsal of other, since St. Paul alone shall serve my turn herein, as well as if I alleged ye a thousand. Yea and if David in the war against the Philistines was reputed as good as ten thousand, well may St. Paul perdy, for the proof of that we now speak of, in the fight for the faith against the faithless persecutors, be accounted as sufficient as if I rehearsed ye ten thousand beside. Then this most valiant champion St. Paul, which was so ravished with the love of Christ and the hope he had in him, that he reckoned himself assured of his reward in heaven, insomuch that he said: \textscripture[2 Ti 4:7]{I have fought a good battle, my course have I finished, my faith have I kept, in time coming have I a crown of justice reserved for me,} which he so fervently desired and longed for, that he spake these words of himself: \textscripture[Pp 1:21]{Christ is my life, and to die were mine advantage,} and again: \textscripture[Pp 1:23]{I long to be discharged of this body of mine, and to be with Christ,} this selfsame Paul I say for all this, both by policy procured to escape the Jews' deceitful trains through the help of a certain captain of the Roman garrison, and afterward got out of prison, alleging that he was a citizen of Rome, and at another time saved himself from the cruel Jews by appealing unto Caesar, and before that, was let down over a wall in a basket, and so avoided the cursed hands of King Aretas.
	
	Here if any man will say that all this while he was in no dread of death at all, but did all this only upon consideration of the great increase of faith, that through his labour and travail might after grow to the world, surely for my part, as I would be loath to deny the one, so dare I not be so bold to warrant the other, since of his own fear that he some time was in (as stronghearted as he was) maketh he sufficient declaration himself where he writeth unto the Corinthians as followeth: \textscripture[2 Co 7:5]{When we came to Macedonia our body had no rest, but much tribulation abode we, battle without and fear within.} Also in another place he writeth unto them in this wise: \textscripture[1 Co 2:3]{In much weakness was I, in sore dread and fear among you.} And again he saith: \textscripture[2 Co 1:8-9]{Brethren, we would not have you ignorant of our trouble which hath happened in Asia, where we have been above our power so afflicted that we were even weary of our life.}
	
	Dost thou not hear now St. Paul with his own mouth confess here his own fear and dread and	wonderful weariness, more intolerable unto him than death. Insomuch that he seemeth by these words, as it were in a painted table, lively to set forth the painful agony he then abode for Christ. Let me now see whether any man can justly say that Christ's holy martyrs were never afraid of death. And yet for all that could no fear cause St. Paul once to shrink or go back from his good purpose to advance the faith of Christ, nor all the counsel the disciples gave him could not stay him, but that needs forth would he to Jerusalem still, as to the place whereunto he saw well that the spirit of God called him, albeit the prophet Agabus had foreshewed him plain, that there should he be both imprisoned, and further in no little danger of his life too.\footnote{Acts 21:10-11}
	
	Whereby it appeareth that to fear death and torment is none offence, but a great and grievous pain, which Christ came not to avoid but patiently to suffer. And we may not by and by judge it a point of cowardice if we see a man either afraid and loath to be tormented, or discreetly eschew peril in such case as he may lawfully do it.
	
	But as I was about to say, it liked Christ of his wonderful goodness thus to do, upon sundry considerations. First because he would fulfil the thing for which he came into this world, and that was to set forth and testify the truth. For whereas he was verily both God and man too, yet some were there which because they considered that he had in him hunger, thirst, sleep, weariness, and suchlike dispositions, as all other men naturally have, falsely mistook him, and believed he was not God indeed. I mean this not only of the Jews and gentiles in his own time that were so much his enemies, but of those Jews and gentiles also which were many years after, that nevertheless called themselves good faithful Christian men; as Arius and the heretics of his sect, who letted not to deny that Christ was one in substance with his Father. Whereby raised they many years together much business and ruffle in the Church. But for a most strong treacle against these venomous heresies, wrought our saviour many a marvellous miracle.
	
	Howbeit afterward rose there as great danger on the other side, as often times from one dangerous peril folk straightways fall in another as jeopardous as the first. For there lacked not some, that so earnestly beheld his glorious and mighty miracles, that the bright shining thereof made their eyes so to dazzle, that contrary to all truth they plainly denied his manhood. Now did these wretches too, following his trade that first began this heresy, never cease by sedition maliciously to break the godly unity of the holy Catholic Church, who by that fond frantic opinion, no less perilous than false, as much as in them lay, laboured to destroy and overthrow the whole mystery of man's redemption, in that they went about to cut from us and, as a man might say, utterly to dry up the gracious moisture of our saviour's death and passion, from whence as out of a well spring issues the water of our salvation.
	
	Now to remedy this deadly disease, it pleased our most gracious and loving physician, by these evident tokens of man's frail nature, as heaviness, fear, weariness, and dread of pain and torment, to declare himself to be a very natural man. Further, for as much as the cause of his coming hither was to suffer sorrow and pain for us, thereby to procure us joy and pleasure, like as the joy that he obtained for us was such as should be to our full contentation in soul and body both, so liked it him not in his body only to endure most cruel tormentry, but inwardly also to feel in his blessed soul the sore anguish of sorrow, fear, and weariness; partly to the end that the more pains he took from us, the more should we be bounden to love him; and partly to put us in remembrance how unreasonable a thing it were, if we should either refuse to abide any trouble and grief for his sake, that willingly abode so many and great for ours, or grudge to take at his hands such punishment as our offences have righteously deserved; considering we here see that our saviour Christ himself, of his own mere goodness, shrank neither in body nor in soul patiently to suffer so manifold and grievous torments, for no desert on his behalf, but only to purge and put away vile and sinful wretchedness.
	
	Finally, likewise as nothing was to him unknown from the beginning, so foresaw he well that there were like to spring up in his mystical body the Church members of divers conditions and qualities. And albeit that to suffer martyrdom nature is not able without the help of grace, since \textscripture{no man,} as saith the apostle, \textscripture[1 Co 12:3]{can say so much as our Lord Jesus but in the spirit of God,} yet doth God in such sort bestow his grace upon mankind, that he letteth not therewhiles nature to work and have her course too, but either suffereth he nature to help forward the grace that he sendeth unto man, to the intent he may the more easily work and do well, or if nature be so froward that it will needs strive there against, yet when it is mastered and overcome by grace, it liketh him that of the difficulty that such folk have in their well doing, there shall grow unto them more matter of merit.
	
	Wherefore forasmuch as Christ did foresee that many there would be so tender of body, that were they never so little in danger of bodily harm, they would be ready forthwith fearfully to tremble and quake, now lest such persons should conceive any inward discomfort, when they should feel themselves so fearful and fainthearted, and see the martyrs again so stout and courageous, and upon fear to be enforced to faint and give over might mishap wilfully to yield and not go through, Christ vouchsafed therefore, I say, to comfort their weak spirits with the example of his own sorrow, heaviness, weariness, and incomparable fear, and unto one that were likely to be in such case, as it were by the lively voice of the precedent, he shewed himself expressly to say: ‘Pluck up thy courage, faint heart, and despair never a deal. What though thou be fearful, sorry, and weary, and standest in great dread of most painful tormentry that is like to fall upon thee, be of good comfort for all that, for I myself have vanquished the whole world, and yet felt I far more fear, sorrow, weariness, and much more inward anguish too, when I considered my most bitter painful passion to press so fast upon me. He that is stronghearted may find a thousand glorious valiant martyrs, whose example he may right joyfully follow. But thou now, O timorous and weak silly sheep, think it sufficient for thee, only to walk after me, which am thy shepherd and governor, and so mistrust thyself and put thy trust in me. For this self same dreadful passage lo! have I myself passed before thee. Take hold on the hem of my garment therefore. From thence shall thou perceive such strength and relief to proceed, as shall much to thy comfort stay and repress this fond fantasy of thine, that maketh thee thus causeless to fear, and give thee better courage, when thou shalt remember, not only that thou followest my steps therein (which am faithful, and will not suffer thee to be tempted above thy power, but give thee with thy temptation a way out, that thou mayest be able to abide it),\footnote{1 Corinthians 10:13} but also that this small and short trouble, which thou sufferest here, shall win thee exceeding great glory in heaven. For the afflictions of this world be nothing worthy the glory that is to come, which shall be revealed in thee.\footnote{Romans 8:18} Now having all these things imprinted in thy remembrance take a good heart unto thee, and with the sign of my cross clearly drive from thee these fearful, heavy, dreadful and dull vain imaginations that the spirit of darkness thus worketh in thee, and prosperously go forward on thy journey, and pass through all trouble and adversity, faithfully trusting that by mine aid and help, thou shalt have the upper hand and of me receive for thy reward the glorious crown of victory.’
	
	Thus among other causes for which our saviour vouchsafed to take upon him these afflictions of our frail nature, one was this which I have here before rehearsed, and that as it seemeth very reasonable, that is to wit, he became weak for their sakes that were weak, by his weakness to cure theirs, whom he so entirely tendered, that in all that ever he did in this his bitter agony, it appeareth he meant nothing more, than to teach the fainthearted soldier how to behave himself in his troublous travail, when he shall be violently drawn to martyrdom. For to the intent he would instruct him that is in fear of danger, both to desire other folk to watch and pray for him, and therewith nevertheless in his own person to recommend himself wholly unto God, and again for that he would have it known that none but himself alone as then should taste the painful pangs of death, when he had commanded those three apostles, whom he took forth with him from the other eight almost to the foot of the hill, to stay still there and to abide and watch with him, then got he himself from them a stone's cast further.
	
	\chapter{A Meditation on Distractedness during Prayer}
	
	\begin{scripture}[Mt 26:39, Mk 14:35-36]
		So when he was gone a little further, down fell he prostrate upon the ground, and prayed, that if it were possible, that hour might pass away from him. And thus he said: O Father, Father, unto thee are all things possible. Take away this cup from me, but yet thy will be fulfilled and not mine. O my good Father, if it may be, let this cup pass from me, howbeit do not as I will herein, but as it liketh thee.
	\end{scripture}
	
	\vspace{10mm}
	
	\lettrine{H}{ere} doth Christ like a good captain teach his soldier by his own example, first of all to begin with humility, the foundation and ground of all other virtues, which once laid, a man may without danger climb up higher. For Christ albeit he was very God, equal and one in substance with God his Father, nevertheless for that he was man also, letted not in most humble wise to cast himself down flat upon the ground before him.
	
	But here, good reader, let us pause awhile, and with entire devotion consider with what meekness our captain Christ lieth thus prostrate upon the ground. For if we earnestly so do, we shall have our hearts so lightened with the bright shining beam of that light, that illumineth every man which cometh into this world, that we shall be able thereby to see, know, lament, and at length to reform this foul folly. For negligent or slothful sluggishness can I not call it, but rather frantic madness and insensible deadly dullness, which causeth a great many of us when we go to make our prayer unto Almighty God, not with reverence attentively to pray to him, but like careless and sleepy wretches hoverly to talk with him. Wherefore I much fear me lest we rather sorely provoke his wrath and indignation, than purchase at his hand any favour or mercy toward us.
	
	Would God we would sometime take so much pain, as soon as we have finished our prayers, as forthwith orderly to call to our remembrance again all things that have passed us in the while we seemed to pray. Lord, how foolish, how fond, and how filthy matters shall we many times there find? We would, I assure you, wonder how our mind could possibly in so short a space stray so much abroad into so many places so far severed asunder, and about so divers and sundry, so many and idle occupations. For if a man would even of purpose for a proof do his endeavour to occupy his thought upon as many and as manifold matters, as by any possibility he could devise, hardly could he, I trow, in so little a while think upon so many things, and so far distant asunder as our idle unoccupied mind wandereth about, while our tongue at adventure pattereth apace, upon our matins and evensong, and other accustomed prayers.
	
	
	And therefore if a body would muse and marvel what our wits are busied withal, when we be troubled with dreams in our sleep, nothing know I whereunto I may better liken our mind for the while, than if we do imagine it to be in like sort occupied while we be sleeping, as it is when we pray waking (if at the least wise he that prayeth after this manner may be counted waking), while we suffer our foolish mad brain in the mean season, so fast to wander about hither and thither upon so sundry fond fantasies. Saving this only difference is there betwixt them, that these which, as a man might say, thus dream waking, have certain so monstrous, so shameful, and so abominable toys in their heads, while their tongue mumbleth up their prayers in haste without any heed taken thereunto, and their hearts be straying abroad therewhiles in other places, that if a man had seen the like but in his sleep, yet even among children would he not, I am sure, for shame (were he never so shameless) at his uprising utter so frantic fantastical dreams
	
	And out of all doubt most true is the old said saw, that the outward behaviour and continuance is a plain express mirror or image of the mind, inasmuch as by the eyes, by the cheeks, by the eyelids, by the brows, by the hands, by the feet, and finally by the gesture of the whole body, right well appeareth how madly and fondly the mind is set and disposed. For as we little pass how small devotion of heart we come to pray withal, so do we little pass also how undevoutly we go forward therein. And albeit we would have it seem that on the holy days we go more gorgeously apparelled than at other times only for the honour of God, yet the negligent fashion that we use a great many of us, in the time of our prayer, doth sufficiently declare (be we never so loath to have it so known and apparent to the world), that we do it altogether of a peevish worldly pride. So carelessly do we even in the church somewhiles solemnly jet to and fro, and other whiles fair and softly set us down again. And if it hap us to kneel, then either do we kneel upon the one knee, and lean upon the other, or else will we have a cushion laid under them both, yea and sometime (namely if we be anything nice and fine) we call for a cushion to bear up our elbows too, and so like an old rotten ruinous house, we be fain therewith to be stayed and underpropped. And then further do we every way discover, how far wide our mind is wandering from God. We claw our head, we pare our nails, we pick our nose and say therewhiles one thing for another, since what is said or what is unsaid both having clean forgotten, we be fain at all adventures to aim what we have more to say. Be we not ashamed, thus madly demeaning ourselves both secretly in our heart, and also in our doings openly, in such wise to sue for succour unto God, being in so great danger as we be; and in such wise to pray for pardon of so many horrible offences; and over that in such wise to desire him to preserve us from perpetual damnation? So that this one offence so unreverently to approach to the high majesty of God, all had we never offended him before, were yet alone well worthy to be punished with a thousand endless deaths.
	
	Well now suppose that thou hadst committed treason against some mighty worldly prince, which were at his liberty either to kill thee or save thee, and this notwithstanding that he would be so merciful unto thee, as upon thy repentance and humble suit for his gracious favour again, be content favourably to change the punishment of death into some fine and payment of money, or further upon the effectual proof and declaration of thine hearty and exceeding shame and sorrow for thy fault, clearly release thee of altogether. Now when thou comest in presence of this prince, suppose thou wouldst unreverently, as one that carelessly passed not what he did, tell thy tale unto him, and while he sat still and gave good ear unto thee, in the uttering of thy suit all the while jet up and down before him, and when thou hadst jetted thy fill squat thee down fair and well in a chair, or if for good manners' sake thou thoughtest it most seemly for thee to kneel on thy knees, yet then that thou wouldst call somebody first, to fetch thee a cushion to lay underneath thee, yea and besides that to bring thee a stool and another cushion therewithal to lean thine elbows on, and after all this gape, stretch, sneeze, spit, thou carest not how, balk out the stinking savour of thy ravenous surfeiting, and finally so behave thyself in thy countenance, speech, gesture, and thy whole body beside, that he might plainly perceive that while thou spakest unto him, thy mind were otherwise occupied; tell me now, I beseech thee, what good, trowest thou, shouldst thou get at his hand by this tale thus told afore him?
	
	
	If we should thus handle a case of life and death, in the presence but of some worldly prince, we would I am sure reckon ourselves even quite out of our wits. Whereas he, when he had killed the body, had done his uttermost, and were able to do no more. And be we then, ween you, well advised, which being found faulty in a great many of matters of much more importance, presume so without reverence to sue for pardon unto the king of all kings, Almighty God himself who, when he hath killed the body, hath power also to cast the soul and body both into the fire of hell for ever.\footnote{Matthew 10:28}
	
	Howbeit I would not any man should so understand my words here, as though I would have nobody to pray either walking or sitting, or lying in his bed either. For gladly would I wish, that whatsoever the body be doing, we would yet in the meanwhile ever more lift up our hearts to God, which is a kind of prayer that he doth most accept, since which way soever we walk, so that our mind be fixed on God never depart we from him which is everywhere present with us. Howbeit like as the prophet that said unto God: \textscripture[Ps 62:7]{I forgat thee not, while I lay in my bed,} did not so satisfy himself therewith, but that he would needs rise at midnight\footnote{Psalms 118:62} too, for to laud and praise our Lord, so beside these prayers that we say thus walking, some yet would I have sometimes in such wise to be said, that both should our minds with so godly meditation be prepared, and our bodies in so reverent manner disposed and ordered, that we could not in more humble wise use ourselves, if we should go unto the princes of the whole world, all were they sitting in one place altogether at once.
	
	And without fail this wandering of the mind, as oft as I bethink me thereupon, troubleth my heart full sore. Yet will I not say that every thought (albeit right shameful and horrible) which in the time of our prayer either is put into our mind by the suggestion of our evil angel, or otherwise by the imagination of our own senses creepeth covertly into us, is forthwith deadly sin, if so be we do resist it and quickly cast it off. But Mary, if we be content either gladly to take in such evil thoughts, or suffer them long carelessly to increase in us, I nothing doubt at all but that the weight thereof may in conclusion grow to very deadly and damnable sin indeed. Moreover, when I consider the high majesty of Almighty God, I must needs straightways deem and believe that albeit to have the mind never so little awhile wandering upon other things, is not accounted for mortal sin, yet proceedeth that rather of God's marvellous mercy towards us, whereby it pleaseth him not so to lay to our charge, than that the thing is not of itself so evil as to deserve damnation, since I can in no wise devise how any such lewd thoughts could possibly enter into men's minds while they be praying, that is to wit, while they be talking with God, but only by means of a faint and feeble faith. For seeing our heart strayeth never a deal when we have communication in an earnest matter with a worldly prince, yea or with any officer of his either, that beareth any stroke about him, it were not possible that we should have so much as one vain and strange fantasy in our heads at all, while we make our prayer unto God, if we did firmly and surely believe, that he were presently with us himself, and not only heard what we say, and marked our outward manner as well in our countenance as in all our other gestures beside, and thereby guessed how our heart were inwardly occupied, but also clearly saw and beheld the very bottom of our stomach, as he that by the infinite brightness of his divine majesty maketh all things lightsome; if we believed, I say, that God himself were present, in whose glorious presence all the princes upon earth, even in their most royalty, must needs (but if they be stark mad) plainly grant themselves to be no better than very vile wretched worms of the earth.
	
	Wherefore our saviour Christ, forasmuch as he perceived that there is nothing more profitable for man than prayer, and therewith again considered that, partly by man's negligence, and partly by the malice of the devil, so wholesome a thing almost everywhere taketh but little effect, yea and ofttimes too doth great hurt and harm, determined while he was going toward his passion, both by the manner of his own prayer, and his own example joined thereto, to set forth so necessary a point, to be as it were a full conclusion of all the rest of his doctrine.
	
	And therefore to give us warning, that we ought not only secretly with our heart, but also with our body openly in the face of the world, to serve and honour God, the creator of them both, and to teach us over this, that the reverent and seemly behaviour of the body, albeit the same principally proceedeth of the fervent devotion of the heart, doth nevertheless cause again our inward fervour and reverence to godward to increase and grow greater, he shewed us then a sample himself of most humble submission in prayer; who with such lowly outward gesture worshipped his heavenly Father, as none earthly prince (unless it were Alexander, when he was in his drunken and riotous rages, and certain other barbarous princes that were so proud of their estate that they looked to have been reputed for gods) durst either for shame require of his subjects, or receive when it was willingly offered. For all the while he prayed, neither did he sit at his ease, nor stand upon his feet, nor yet only kneeled neither, but fell down grovelling flat upon the ground, and there so lying lamentably, besought his Father to be merciful unto him, and still saying: ‘Father, Father,’\footnote{Mark 14:36} humbly desired that he unto whom nothing is impossible, would vouchsafe, if it might so be, that is to wit, unless he had fully determined to have him taste the cup of this painful passion, else at his request and prayer to preserve him from it, being nevertheless content that his request herein should take no place, if unto his blessed will it seemed not so convenient.
	
	We may not by occasion of these words, reckon that the Son was ignorant of his Father's will and pleasure, but as he came hither to instruct and teach men, so would he have it appear unto them, that he had in himself very man's affections. And whereas he said twice: ‘Father, Father,’ he willed us thereby to understand, that God, his Father, is indeed the father of all things both in heaven and earth. Furthermore, he put us by the same in remembrance, that God the Father was to him a double Father. Once by creation, which is a kind of fatherhood, since of truth more rightly come we of God that made us of nought, than of the man that naturally begat us, in as much as God both created our natural father and orderly made and disposed all that matter whereof we ourselves are engendered. And albeit Christ as man in this wise took God for his Father, yet as God took he him for his natural and coeternal Father.
	
	It may well be too, that he twice called upon him by this name Father, to have it known that he was not alonely a natural Father unto him in heaven but also that he had none other Father here in the world neither, for as much as he was conceived in his manhood of his mother, being a pure virgin without man's seed, by the coming of the Holy Ghost that entered into his mother, that Holy Spirit, I mean, which proceedeth both from his Father and himself, whose doings be evermore all one, and can in no wise by any man's imagination be dissevered.
	
	Now by this his so oft and earnest calling him Father, which declareth an effectual desire to obtain his request, we learn another wholesome lesson beside: that whensoever we heartily pray for anything, and do not forthwith speed thereof, we should not faint and be utterly therewith discouraged, as was the wicked king Saul, who because he received not an answer from God by and by as he looked for, sought unto a witch and so fell to sorcery and witchcraft, which was both by God's law forbidden, and by himself also not long before inhibited. Then hereby doth Christ teach us still to persevere in prayer, and although we do never obtain the thing which we require, that yet we should not repine and grudge thereat, considering that, as we see here, the Son of God our saviour himself did not obtain his own delivery from death, which he most instantly prayed unto his Father for, saving that evermore (in which part specially ought we to follow his example) he submitted and conformed his own will to the will of his Father.
	
	\textscripture[Mt 26:40, Mk 14:16]{And he came to his disciples, and found them asleep.} Here may we see what difference there is in love. For that love, lo! that Christ bare unto his disciples, very far surmounted the love that they bare toward him again, even they I say that loved him best of all. Who for all the sorrow, fear, dread, and weariness he was so sore panged with, his most bitter passion drawing so fast upon him, could not for all that forbear, but that needs would he even then go and see how they did; whereas they on the other side, how great love so ever they bare him, as without fail they loved him full tenderly, for all the exceeding peril they saw their most loved master so likely forthwith to fall in, were yet never the more able to keep themselves from sleep.
	
	
	\chapter{A Meditation on Keeping Watch}

	\begin{scripture}[Mt 26:40-41, Mk 14:37-38]
		Then said he thus to Peter: Sleepest thou, Simon? Couldst thou not endure to watch one hour with me? Watch and pray that ye enter not into temptation. The spirit is prompt and ready, but the flesh is frail and weak.
	\end{scripture}
	
	\vspace{10mm}
	
	\lettrine{O}{h!} what force and efficacy is there in these few words of Christ! And in these gentle words of his, Lord, how sharply doth he touch him! For in that he called him here by the name of Simon, and so called him when he laid to his charge his sluggish sleeping, thereby did he secretly signify that such feebleness and slothful sluggishness was full unfit for him that bare the name of Peter, which name Christ for his constant steadfastness he would should have been in him, had given long erst105 unto him. And as it was a privy check unto him that he called him not by the name of Peter or Cephas, so sounded it again to his reproach that he named him Simon. For in the Hebrew tongue in which Christ at the same time spake unto him, Simon is as much to say as hearing and obedient. But now when he contrary to Christ's admonition fell to sleeping, then did he neither hear Christ nor obey him neither.
	
	And yet as me seemeth did our saviour not in this wise only covertly control Peter by these his mild words unto him, but somewhat sharply nipped him otherwise also, as if he had earnestly thus spoken unto him and said: ‘What, Simon, here playest thou not the part of Cephas, for why shouldst thou any more be called Cephas, that is to wit, a stone, which name I gave thee heretofore to have thee steadfast and strong, when thou shewest thyself so feeble and faint now sleep cometh on thee, that thou canst not abide to watch so much as one hour with me. What, Simon, I say, art thou now fallen asleep? And well worthy art thou perdy to be called by thy first name Simon, for since thou art so heavy asleep, how shouldst thou be named Simon, that is to say, a hearer? Or seeing that I warned thee to watch with me, how canst thou be called obedient? Which as soon as my back was turned like a slothful sluggard straightways wert fallen asleep. Simon, I evermore made most of thee, and art thou now asleep? Simon, I have so many ways advanced thee, and dost thou now sleep? Simon, thou didst but right now boldly boast that if need were thou wouldst die with me, and dost thou now sleep? Simon, even at this point do the Jews and gentiles, and Judas worse than either of them, go about to murder me, and yet dost thou sleep? Yea, Simon, and the devil too laboureth to sift ye all like wheat, and art thou still asleep? Oh! what may I reckon that the rest of my disciples will do, when thou, Simon, seeing me and yourselves too in so extreme peril, art now thus fallen asleep?’
	
	After these words because it should not seem that he touched Peter alone, he began to say unto the rest also: \textscripture{Watch and pray, that ye enter not into temptation. The spirit is prompt and ready, but the flesh is frail and weak.} Here are we warned continually to pray and here are we taught how profitable and very needful prayer is to stay us, that our frail flesh do not draw back and stop our welldisposed heart, and train it headlong into dangerous deadly temptation. For who was bolder spirited than Peter? And yet how greatly he needed the aid of God to assist him against his frail flesh, plainly appeareth by this, that while by his sleeping he forslothed108 to pray and call for God's help, he gave the devil such advantage upon him, that through the feebleness of his flesh, his courageous spirit was soon after abated, and himself driven clearly to deny and forswear Christ.
	
	Now if it thus fared with the apostles, being so fresh and forward, that while through sleeping they discontinued their prayer they fell into temptation, what shall become of us withered and barren wretches, if in time of danger (which, God wot, seldom are we out of, since our adversary the devil like a ramping lion runneth evermore about, everywhere seeking whom by frailty fallen into sin he may forthwith catch and devour)\footnote{1 Peter 5:18} what shall become of us, I say, if in such danger we do not, as Christ bade us, persevere in watch and prayer? Here Christ biddeth us watch, not to play at cards and dice, not to banquet and surfeit, not to drink ourselves drunk, and fulfil our filthy lusts, but he biddeth us watch to pray. And pray doth he bid us not now and then among, but always without any ceasing. \textscripture{Pray ye,} saith he, \textscripture[Lk 18:1, 1 Th 5:17]{without intermission.} And he would have us pray, not in the daytime only (for who would bid anybody to watch in the day), but he admonisheth us to bestow also even a good part of that time in hearty prayer that a great sort of us are wont to spend altogether in sleep. Wherefore ought we wretched caitiffs much to be ashamed of ourselves and to acknowledge how grievously we do offend, which scantly in the day say any short prayer at all, and yet as short as it is, full slightly cometh it from us, and as though we were half asleep.
	
	Finally our saviour willeth us to pray, not for abundance of riches, and plenty of other worldly pleasures, nor to have hurt light on our enemies, nor to receive honour here in this world, but that we fall not into temptation; willing us therein to understand that all those worldly things be either very perilous and hurtful or else, in comparison of this one thing, very vain and foolish trifles. And therefore that thing, as the principal point that briefly implieth all the rest, did he purposely place in the end of that prayer, which long before he had taught his disciples, where he willed them to pray thus: \textscripture[Mt 6:13]{And suffer us not to be led into temptation, but deliver us from evil.}
	
\end{document}
