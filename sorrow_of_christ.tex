\documentclass[a5paper]{scrbook}

% Disable left-right margin on page turn
\KOMAoptions{twoside=false}

% Util Package
\usepackage{bimscommands}
% Scripture passages
\usepackage{scripture}
% Drop Caps
\usepackage{Carrickc,lettrine}
\renewcommand\LettrineFontHook{\Carrickcfamily}

\begin{document}
	\title{The Sorrow of Christ}
	\subtitle{In English}
	\author{St. Thomas More}
	\frontmatter
	\maketitle
	\tableofcontents
	\mainmatter
	\bimspart{Of The Sorrow, Weariness, Fear, And Prayer Of Christ Before His Taking}{As it is written in the XXVI Chapter of St. Matthew, The XIII of St. Mark, The XXII of St. Luke, and the XVIII of St. John.}
	\chapter{A Meditation on Grace and Olivet}
	
	\begin{scripture}[Matthew 26:30]
		\vs{30}When Jesus had spoken these words, and said grace, they went forth into the
		Mount of Olivet.
	\end{scripture}
	
	\vspace{10mm}
	
	\lettrine{A}{lbeit} that Christ at the time of his supper had had so much godly communication with his apostles, yet forgot he not at his departing to make an end of all together, with thanksgiving to God. But how unlike, alas! be we to Christ, which bear the name of Christian men, and yet at our table do use, not only many vain and idle words (whereof Christ hath given us warning that we shall yield a full strait account), but also very hurtful and perilous, and at last when we have eaten and drunk our fill, unkindly get us our way, forgetting to give thanks unto God the giver of all, that hath so well fed and refreshed us.
	
	Burgensis\footnote{Paul of Burgos (c.1351–1435), a Jewish convert, who later became Archbishop of said Burgos.}, a man well learned and deeply travailed in divinity, upon probable conjectures doth think that the grace, which Christ at the same time said with his apostles, was those six psalms which, as they stand together, the Hebrews call the great Alleluia: that is to wit, the hundredth and twelfth psalm with the five next following in order. For those six psalms, which they name the great Alleluia, they were wont of an old custom to say instead of grace at Easter and certain other high feasts. And the selfsame grace as yet to this day at the said feasts commonly use they to say.
	
	But as for us, whereas we have been accustomed in times past, for grace both before meat and after, to say at sundry seasons sundry psalms such as be most convenient for the time, we have nowadays given them over almost every one, so that with three or four words, whatsoever suddenly cometh to our minds, and those overly mumbled up at adventure, shortly make we an end and depart.
	
\end{document}
