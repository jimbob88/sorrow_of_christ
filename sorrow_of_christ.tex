\documentclass[a5paper]{scrbook}

% Disable left-right margin on page turn
\KOMAoptions{twoside=false}

% Util Package
\usepackage{bimscommands}
% EB Garamond Font for Holy Writ
\usepackage[oldstyle]{ebgaramond}
\usepackage{fontenc}
% Scripture passages
\usepackage{scripture}
\scripturesetup{font=\ebgaramond}
% Drop Caps
\usepackage{Carrickc,lettrine}
\renewcommand\LettrineFontHook{\Carrickcfamily}

\begin{document}
	\setmainfont{Times}
	\title{The Sorrow of Christ}
	\subtitle{In Old Fashioned English}
	\author{St. Thomas More}
	\frontmatter
	\maketitle
	\tableofcontents
	\mainmatter
	\bimspart{Of The Sorrow, Weariness, Fear, And Prayer Of Christ Before His Taking}{As it is written in the XXVI Chapter of St. Matthew, The XIII of St. Mark, The XXII of St. Luke, and the XVIII of St. John.}
	\chapter{A Meditation on Grace and Olivet}
	
	\begin{scripture}[Matthew 26:30]
		\vs{30}When Jesus had spoken these words, and said grace, they went forth into the
		Mount of Olivet.
	\end{scripture}
	
	\vspace{10mm}
	
	\lettrine{A}{lbeit} that Christ at the time of his supper had had so much godly communication with his apostles, yet forgot he not at his departing to make an end of all together, with thanksgiving to God. But how unlike, alas! be we to Christ, which bear the name of Christian men, and yet at our table do use, not only many vain and idle words (whereof Christ hath given us warning that we shall yield a full strait account), but also very hurtful and perilous, and at last when we have eaten and drunk our fill, unkindly get us our way, forgetting to give thanks unto God the giver of all, that hath so well fed and refreshed us.
	
	Burgensis\footnote{Paul of Burgos (c.1351–1435), a Jewish convert, who later became Archbishop of said Burgos.}, a man well learned and deeply travailed in divinity, upon probable conjectures doth think that the grace, which Christ at the same time said with his apostles, was those six psalms which, as they stand together, the Hebrews call the great Alleluia: that is to wit, the hundredth and twelfth psalm with the five next following in order. For those six psalms, which they name the great Alleluia, they were wont of an old custom to say instead of grace at Easter and certain other high feasts. And the selfsame grace as yet to this day at the said feasts commonly use they to say.
	
	But as for us, whereas we have been accustomed in times past, for grace both before meat and after, to say at sundry seasons sundry psalms such as be most convenient for the time, we have nowadays given them over almost every one, so that with three or four words, whatsoever suddenly cometh to our minds, and those overly mumbled up at adventure, shortly make we an end and depart.
	
	\textscripture[Mt 26:30]{They went forth unto the mount of Olivet.} Forth they went, but not to bed. \textscripture{I rose at midnight,} saith the prophet, \textscripture[Ps 118:62]{to give praise and thanks to thee.} Howbeit Christ did not so much as once lay him down on his bed. But at the leastwise, would God we could truly say: \textscripture[Ps 62:7]{I remembered thee in my bed, good Lord.}
	
	And it was not in the summer season neither that Christ after his supper took his way to the mount. For it was even shortly after the spring of the year, when the days and the nights be all of one length. And that it was a cold night appeareth also by this, that the servants were warming themselves by the fire in the bishop's hall. And that this was not the first time that he so did, well witnesseth the evangelist where he saith: \textscripture{According to his custom.}
	
	He went up to the mount to pray, willing us thereby to understand that when we set ourselves to pray, we must lift up our hearts from the cumbrous unquietness of all worldly business, to the end we may wholly set our minds upon God and godly matters.
	
	This mount of Olivet which was all full of olive trees, containeth in it a certain mystery. For a branch of an olive tree was commonly taken as a token of peace, which Christ came himself to make betwixt God and man, who had so long before been enemies.
	
	Besides this, the oil that cometh of the olive tree doth signify the grace of the Holy Ghost, whom Christ did come to send down to his disciples after his return to his Father: to the end that by the grace of the same Holy Spirit, they might within short space after be able to learn those things which, if he had told them then, they could not well have borne.
	
	\chapter{A Meditation on Cedron}
	
	\begin{scripture}[Jn 18:1, Mt 26:36, Mk 14:32]
		Over a river called Cedron into a village which is named Gethsemani.
	\end{scripture}
	
	\vspace{10mm}
	
	\lettrine{T}{his} river Cedron runneth between the city of Jerusalem and the mount of Olives. And this word Cedron, in the Hebrew tongue, signifieth sorrow or heaviness. And Gethsemani in the same speech is as much to say as a very fat and plentiful valley, or otherwise the valley of Olivet.
	
	We have therefore good cause to think that the evangelists not without great consideration did so diligently rehearse the names of these places, for else they would have thought it sufficient to have shewed that he went forth unto the mount of Olivet, had it not been that God, under the names of those places, had secretly covered some high mysteries, which, by the rehearsal of those names, good men and studious should have occasion afterwards, through the aid of his Holy Spirit, to search out.
	
	For since we may in no wise think that there is any superfluous syllable in the sacred scripture, which the apostles wrote by the inspiration of the Holy Ghost, and that not so much as a sparrow lighteth upon the ground without the will of God\footnote{Matthew 10:29}, I must needs believe that neither the evangelists made mention of those names without some good cause, nor yet that the Hebrews so named them (whatsoever their purpose was when they did so call them), but by some secret motion (albeit to themselves unknown) of God's own Holy Spirit, which under those names had closely hid certain notable mysteries, and at length should be brought to light.
	
	And since Cedron signifieth sorrow and blackness too, and besides that is the name, not of the river only which the evangelists do here make mention of, but also, as we may well perceive, of the valley that the river passeth through, which valley lieth betwixt Jerusalem and Gethsemani, these names (but if we be too slothful and negligent) do put us in remembrance that as long as we live here (as the apostle with), like strangers sequestered from our Lord\footnote{2 Corinthians 5:6}, we must needs pass over, ere ever we come unto the fruitful mount of Olivet, and the pleasant village of Gethsemani (a village, I say, not displeasant or loathsome to look upon, but full of all delight and pleasure), we must first pass over, as I said, this valley and river called Cedron, a vale of misery and river of heaviness, the water whereof may clean, purge, and wash away, the foul black filthiness of our sins.
	
	But now if we, to avoid grief and pain, go about by a contrary way, to make this world, which should be a place of pain and penance, to be a place of ease and pastime, and so turn it unto our heaven, both do we clearly exclude ourselves from the very true felicity for ever, and drown us all too late in fruitless sorrow and care, and further bring ourselves into intolerable and endless wretchedness. And this wholesome lesson are we put in mind of by the wellplaced rehearsal of Cedron and Gethsemani.
	
	Now because the words of holy scripture have not one sense alone, but are full of many mysteries, the names of these places do so well serve to the setting forth of this history of Christ's passion, as though for the same purpose only God had from the beginning ordained those places long before to be called by such notable names, as being compared with those things that Christ did many years after, might declare that they were appointed aforehand to be as it were witnesses of his most bitter passion. For since Cedron signifieth black, doth it not seem to express the saying of the prophet, which was spoken of Christ going to his glorious kingdom by most shameful death, disfigured with stripes, blood, spiteful spitting, and such other filthiness, where it is written: \textscripture[Is 53:2]{Neither comeliness nor beauty is there in him.} And that the river which he passed over did not without cause betoken sorrow and heaviness, himself right well witnessed where he said: \textscripture[Mt 26:38]{My soul is heavy even to the death.}
	
	\chapter{A Meditation on Betrayal}
	
	\begin{scripture}[Luke 22:39]
		And his disciples went with him.
	\end{scripture}
	
	\vspace{10mm}
	
	\lettrine{I}{t} is to be understood of the eleven only which still remained with him. For the twelfth, whom the devil entered into after he had eaten the sop, and carried forth from the residue of the apostles, waited now no longer upon his master as his disciple, but like a traitor laboured to destroy him. And so proved these words of Christ too true: \textscripture[Mt 12:30]{He that is not with me is against me.}
	
	Let us follow Christ therefore, and by prayer call upon his Father with him. And let us not, as Judas did, slip aside from him, after we have been relieved by his gracious goodness, and well and liberally supped with him, for fear this saying of the prophet be verified in us: \textscripture[Ps 49:18]{If thou sawest a thief thou didst run with him, and with adulterers didst thou pay thy shot.}
	
	\chapter{A Meditation on Prayer}
	
	\begin{scripture}[John 18:2]
		And Judas that did go about to betray him, knew right well the place, because Jesus used often times to come thither with his disciples.
	\end{scripture}
	
	\vspace{10mm}
	
	\lettrine{N}{ow} by occasion of the traitor do the evangelists yet once again both beat into us, and with oft rehearsal thereof much commend also, the blessed custom of Christ who was wont to resort thither with his disciples to pray. For if he had not gone to the same place so commonly in the night time, but now and then among, the traitor could not have been so well assured to find our Lord there, that he durst have conducted thither the bishop's servants and a band of the Roman soldiers, as to the thing they should not miss to meet withal; since if they had found it otherwise, they would have went he had mocked them, and so ere he could have escaped away, haply have done him some displeasure.
	
	But now where are these folk become, that stand very much in their own conceit, and as though they had done a great feat, fondly glory in themselves, if it hath fortuned them at one time or other, on high evens, either to watch anything long in prayer by night, or else for the same purpose to rise in the morning somewhat early? Our saviour Christ customably used to persevere in prayer all the whole night without any sleep at all.
	
	Where be they also which, because he refused not to eat and drink with the publicans, nor disdained not to receive kindness and service of sinners, called him a glutton and a drunkard, and in
	comparison of the Pharisees, whose profession was very strait counted him to be scant in virtue so perfect as one of the common sort? And yet while these sour lowering Pharisees, to be seen of the world, were praying openly abroad in corners of the streets, he therewhiles full mildly and lovingly taught sinful men, while he ate and drank with them, to amend their lives. Again while the false dissembling pharisee lay at his ease routing in his soft bed, Christ continued without doors painfully all night in prayer
	
	Oh, would God we which are so slack and slothful that we cannot follow the good example of our saviour in this behalf, would yet at the least wise, when we turn ourselves in our bed even ready to fall asleep, have in remembrance Christ's continual watch, and although it were in few words, till sleep come on us again, give him hearty thanks, both misliking our own sluggishness and therewithal desiring him to endue us with more of his grace. Surely if we would accustom ourselves to do but even so much, I nothing doubt but that God would within short space help us with his grace and make us much better.
	
\end{document}
